\begin{abstract}
Educational simulations have long been recognized as powerful tools for enhancing learning outcomes, yet their creation has traditionally required substantial resources and technical expertise. This paper introduces \textit{MicroSims}---a novel framework for creating lightweight, interactive educational simulations that can be rapidly generated using artificial intelligence, universally embedded across digital learning platforms, and easily customized without programming knowledge. MicroSims occupy a unique position at the intersection of three key innovations: (1) standardized design patterns that enable AI-assisted generation, (2) iframe-based architecture that provides universal embedding and sandboxed security, and (3) transparent, modifiable code that supports customization and pedagogical transparency. We present a comprehensive framework encompassing design principles, technical architecture, metadata standards, and development workflows. Drawing on empirical research from physics education studies and meta-analyses across STEM disciplines, we demonstrate that interactive simulations can improve conceptual understanding by 30-40\% compared to traditional instruction. MicroSims extend these benefits while addressing persistent barriers of cost, technical complexity, and platform dependence. This work has significant implications for educational equity, enabling educators worldwide to create customized, curriculum-aligned simulations on demand. We discuss implementation considerations, present evidence of effectiveness, and outline future directions for AI-powered adaptive learning systems built on the MicroSim foundation.
\end{abstract}
