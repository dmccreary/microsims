\section{MicroSims: Definition and Characteristics}
\label{sec:definition}

% TODO: Extract content from docs/chapters/what-is-a-microsim.md
% Key sections to include:
% - Formal definition of MicroSims
% - Key attributes (browser-based, focused scope, generative AI compatible)
% - Technical architecture overview
% - Educational purpose and learning objectives
% - What MicroSims are NOT (distinctions)
% - Metadata strategy overview

\subsection{Formal Definition}

\textit{Educational MicroSims} are lightweight, browser-based interactive simulations designed to demonstrate specific concepts or principles through direct manipulation and immediate feedback. They occupy a unique position at the intersection of simplicity, accessibility, and AI-generation capability.

\subsection{Core Characteristics}

\textbf{Focused Scope}: MicroSims deliberately constrain their focus to specific learning objectives rather than attempting comprehensive coverage of broad subject domains.

\textbf{Browser-Native}: All MicroSims run entirely in web browsers using HTML5, CSS, and JavaScript, requiring no installation or specialized software.

\textbf{Generative AI Compatible}: MicroSims follow standardized design patterns that enable large language models to generate, modify, and extend them based on natural language descriptions.

\textbf{Universal Embedding}: Through iframe architecture, MicroSims can be embedded in any digital environment that supports web content.

\textbf{Transparent Implementation}: MicroSim code is intentionally readable and well-documented, enabling educators and students to examine, understand, and modify the underlying logic.

% Additional content to be extracted from source files
